\documentclass[11pt]{article}

\usepackage[a4paper]{geometry}
\usepackage[utf8]{inputenc}
\usepackage[english]{babel}

\usepackage{amsmath}
\usepackage{amssymb}

\newcommand{\rmd}{{\mathrm d}}
\newcommand{\rme}{{\mathrm e}}

\renewcommand{\vec}[1]{{\boldsymbol {#1}}}

\begin{document}
\title{Contribution from Stratonovich integral in Ratchet potential}
\author{Jiří Pešek}
\maketitle

The ``mechanical'' work of a diffusing particle caused by presence of a potential $V$ is defined via Stratonovich integral as  
\[
{\mathcal Q} = - \int \rmd \vec{x}_t \circ \vec{\nabla} V(\vec{x}_t,t) .
\]
This mechanical work from the point of view of thermodynamics corresponds to the heat, as it represents the changes in energy associated with the change of internal state. 

In case no additional external force is applied except for the aforementioned potential $V$ the equation of motion of such particle is given by
\[
\rmd \vec{x}_t = - \chi \cdot \vec{\nabla} V( \vec{x}_t, t ) \; \rmd t + \sqrt{2 D} \cdot \rmd \vec{W}_t , 
\]
where the mobility matrix $\chi$ and diffusion matrix $D$ are related by $ k_B T \chi = D$ and $\vec{W}_t$ is the Wiener process. 

This can be used for evaluation of the heat 
\[
{\mathcal Q} = \int \rmd t \; \vec{\nabla} V (\vec{x}_t) \cdot \chi \cdot \vec{\nabla} V(\vec{x}_t) - \int \rmd \vec{W}_t \cdot \sqrt{2 D} \circ \vec{\nabla} V(\vec{x}_t) ,
\]
which can be further simplified by using It\^{o} lemma to 
\begin{equation}
{\mathcal Q} = \underbrace{\int \rmd t \; \vec{\nabla} V (\vec{x}_t) \cdot \chi \cdot \vec{\nabla} V(\vec{x}_t) 
- \int \rmd \vec{W}_t \cdot \sqrt{2 D} \cdot \vec{\nabla} V(\vec{x}_t) }_{ = - \int \rmd \vec{x}_t \cdot \vec{\nabla} V(\vec{x}_t) }
- \int \rmd t \; D : \vec{\nabla}^2 V(\vec{x}_t) ,
\label{equ:heat}
\end{equation}

In a simple 1D case this lead to 
\[
\left\langle {\mathcal Q} \right\rangle = \chi \int \rmd t \; \left\langle \left[ \partial V(x_t) \right]^2 \right\rangle
- D \int \rmd t \; \left\langle \partial^2 V(x_t) \right\rangle ,
\]
which for Ratchet potential can be further evaluated in a steady state:
\begin{multline}
\left\langle {\mathcal Q} \right\rangle = \chi \left[ \frac{ \Delta V^2 }{ \ell^2 a^2 } {\mathbb P}( x_t \in F_+ ) + \frac{ \Delta V^2 }{ \ell^2 (1-a)^2 } {\mathbb P}(x_t \in F_- ) \right] \Delta T 
- \\
- \frac{ D \, \Delta V }{\ell a (1-a) } \left( \rho(x_{V_\text{min}}) - \rho(x_{V_\text{max}}) \right) \Delta T,
\label{equ:mean_heat}
\end{multline}
where $\ell$ is the length of the step, $\Delta V$ is the height of the potential,  $a$ is the steepness of the potentials $+$ slope ( $1-a$ is for $-$ slope), $\Delta T$ is time duration of the process, $\rho$ is the stationary probability density, here at the minimum and maximum of potential and $\mathbb P$ denotes the actual probabilities to be on a corresponding slope ( ${\mathbb P}_+ = 1 - {\mathbb P}_-$ ).

Hence such quantity is extensive in time. 
However in equilibrium the situation simplifies. 

\section{Equilibrium contributions} 
Equilibrium probability density is given by Boltzman distribution
\[
\rho(x) = \frac{1}{Z} \rme^{ - \beta V(x) } 
\]
with the partition function given by
\begin{align*}
Z &= \int\limits_0^{a \ell} \rmd x \; \rme^{ - \beta \Delta V \frac{ a \ell - x }{a \ell } }
+ \int\limits_{a \ell}^\ell \rmd x \; \rme^{ - \beta \Delta V \frac{ x - a \ell }{ (1-a) \ell } } \\
&= \rme^{ - \beta \Delta V } \frac{a \ell}{\beta \Delta V} \int\limits_0^{\beta \Delta V} \rmd u \; \rme^u 
+ \frac{(1 - a) \ell}{\beta \Delta V} \int\limits_0^{\beta \Delta V} \rmd u \; \rme^{-u} \\
&= \frac{\ell}{\beta \Delta V} \left( 1 - \rme^{-\beta \Delta V} \right).
\end{align*}
Consequently we can evaluate the probabilities for left and right slope 
\begin{align*}
{\mathbb P}(x_t \in F_+) &= a, \\
{\mathbb P}(x_t \in F_-) &= 1 - a
\end{align*} 
and probability densities at minimum and maximum
\begin{align*}
\rho(x_{V_\text{min}}) &= \frac{\beta \Delta V}{\ell \left( 1 - \rme^{-\beta \Delta V}\right)}, \\
\rho(x_{V_\text{max}}) &= \frac{\beta \Delta V}{\ell \left( 1 - \rme^{-\beta \Delta V}\right)} \rme^{-\beta \Delta V}. 
\end{align*} 

Hence 
\[
\left\langle {\mathcal Q} \right\rangle = 0.
\]
Notice, that if we instead of Stratonovich use It\^{o} integral, only the first term in \eqref{equ:mean_heat} remains and we get non-zero production of heat in equilibrium 
\[
\left\langle {\mathcal Q}_\text{It\^{o}} \right\rangle = \frac{ \chi \Delta V^2}{\ell^2 a (1-a) } \Delta T. 
\]

\section{Fluctuations} 
In order to determine the fluctuations of the work we need to evaluate $\langle {\mathcal Q}^2 \rangle$. 
Let us plug in \eqref{equ:heat} 
\begin{multline*}
\langle {\mathcal Q}^2 \rangle = \left\langle 
\int \rmd t \; \vec{\nabla} V(\vec{x}_t) \cdot \chi \cdot \vec{\nabla} V(\vec{x}_t)
\int \rmd s \; \vec{\nabla} V(\vec{x}_s) \cdot \chi \cdot \vec{\nabla} V(\vec{x}_s) 
\right\rangle - \\
- 2 \left\langle  
\int \rmd t \; D : \vec{\nabla}^2 V(x_t) 
\int \rmd s \; \vec{\nabla} V(\vec{x}_s) \cdot \chi \cdot \vec{\nabla} V(\vec{x}_s)  
\right\rangle + \\ 
+ \left\langle
\int \rmd t \; D : \vec{\nabla}^2 V(x_t) 
\int \rmd s \; D : \vec{\nabla}^2 V(x_s) 
\right\rangle
+ 2 \left\langle
\int \rmd t \; \vec{\nabla} V(\vec{x}_t) \cdot D \cdot \vec{\nabla} V(\vec{x}_t)
\right\rangle .
\end{multline*}
Although the evaluation of time correlations is usually nontrivial, 
their contribution to the total variance $\langle {\mathcal Q}^2 \rangle - \langle {\mathcal Q} \rangle^2$ is non-negative, hence one can obtain a lower bound for fluctuations given by the last term 
\[
\langle {\mathcal Q}^2 \rangle - \langle {\mathcal Q} \rangle^2 
\ge 
2 \left\langle
\int \rmd t \; \vec{\nabla} V(\vec{x}_t) \cdot D \cdot \vec{\nabla} V(\vec{x}_t)
\right\rangle .
\]

In equilibrium this lower bound can be evaluated 
\[
\langle {\mathcal Q}^2 \rangle - \langle {\mathcal Q} \rangle^2 
\ge 
\frac{ 2 D \Delta V^2 }{\ell^2 a (1-a) } \Delta T. 
\]
Hence the variance is the geometric mean between $2 k_B T$ and the ``heat'' in It\^{o} sense. 

\section{Full system}
Let us assume two particles with ratchet potential interaction $V_t$ between them
\begin{align*}
\rmd x_M &= \frac{1}{\gamma_M} \left[ - \nabla_M V_t(x_M - x_A ) + F \right] \; \rmd t + \sqrt{ \frac{2 K_B T}{\gamma_M} } \cdot \rmd W_M, \\
\rmd x_A &= - \frac{1}{\gamma_A} \nabla_A V_t(x_M - x_A ) \; \rmd t + \sqrt{ \frac{2 K_B T}{\gamma_A} } \cdot \rmd W_A . 
\end{align*}
Moreover we apply a constant force $F$ on the particle representing the motor. 
The dependency on the internal state is hidden in the time-dependency of the ratchet potential. 

The heat produced by the diffusion of the motor is given by 
\[
{\mathcal Q}_M = - \int \nabla_M V_t(x_M - x_A) \circ \rmd x_M , 
\]
which by using the same technique as in previous case we get 
\begin{align*}
{\mathcal Q}_M 
=& - \int \nabla V_t(x_M - x_A) \cdot \rmd x_M - \frac{k_B T}{\gamma_M} \int \Delta V_t(x_M - x_A) \; \rmd t \\ 
=& \frac{1}{\gamma_M} \int \nabla V_t(x_M - x_A) \cdot \left[ \nabla V_t(x_M - x_A) - F \right] \; \rmd t - \sqrt{ \frac{ 2 k_B T }{\gamma_M} } \int \nabla V_t(x_M - x_A) \cdot \rmd W_M - \\
& - \frac{k_B T}{\gamma_M} \int \Delta V_t(x_M - x_A) \; \rmd t . 
\end{align*}
The mean value is then given by
\[
\left\langle {\mathcal Q}_M \right\rangle 
= \frac{1}{\gamma_M} \int \left\langle \nabla V_t(x_M - x_A) \cdot \left[ \nabla V_t(x_M - x_A) - F \right] - k_B T \Delta V_t(x_M - x_A) \right\rangle \; \rmd t . 
\]
Similarly the heat dissipated by actin is given by 
\[
\left\langle {\mathcal Q}_A \right\rangle 
= \frac{1}{\gamma_A} \int \left\langle \left[ \nabla V_t(x_M - x_A) \right]^2 - k_B T \Delta V_t(x_M - x_A) \right\rangle \; \rmd t . 
\]
Hence the total mean dissipated heat
\begin{multline*}
\left\langle {\mathcal Q} \right\rangle 
= \left( \frac{1}{\gamma_A} + \frac{1}{\gamma_M} \right) \int \left\langle \left[ \nabla V_t(x_M - x_A) \right]^2 - k_B T \Delta V_t(x_M - x_A) \right\rangle \; \rmd t - \\
- \frac{F}{\gamma_M} \cdot \int \left\langle \nabla V_t(x_M - x_A) \right\rangle \; \rmd t . 
\end{multline*}
In the steady state the mean dissipated heat correspond to 
\begin{multline*}
\left\langle {\mathcal Q} \right\rangle 
= \left( \frac{1}{\gamma_A} + \frac{1}{\gamma_M} \right) \left\langle \left[ \nabla V_t(x_M - x_A) \right]^2 - k_B T \Delta V_t(x_M - x_A) \right\rangle_\text{ss} \, \Delta T - \\
- \frac{F}{\gamma_M} \cdot \left\langle \nabla V_t(x_M - x_A) \right\rangle_\text{ss} \, \Delta T . 
\end{multline*}

The conclusion is that the dissipation in the case of no external force applied ($F=0$), the mean total heat is split between the motor and actin proportionally to $1/\gamma_M$ and $1/\gamma_A$ respectively. 


\section{Empirical distribution}
Related to previous calculations is the question how to determine the steady state distribution. 
What we are doing in the simulations is basically sampling the empirical distribution $\mu$ out of the actual distribution $\rho$. 

In discretized case, the probability to observe an empirical distribution $\mu$ drawn out of the actual distribution $\rho$ is given by 
\[
{\mathbb P}\left( \left\{ \mu_i \right\} \middle| \left\{ \rho_i \right\} \right) 
= \binom{N}{ N \mu_1 \ldots N \mu_K } \prod\limits_{i=1}^K \rho_i^{N \mu_i} 
\approx \exp \left[ N \sum\limits_{i=1}^K \mu_i \ln \frac{\rho_i}{\mu_i} \right] 
\]
By using a saddle point method, with the constraint $\sum_i \mu_i = 1$, the probability can be further approximated by 
\[
{\mathbb P}\left( \left\{ \mu_i \right\} \middle| \left\{ \rho_i \right\} \right) 
\approx \exp \left[ - \frac{N}{2} \sum\limits_{i=1}^K \frac{ \left( \mu_i - \rho_i \right)^2 }{ \rho_i } \right] .
\]

Hence it is obvious that the mean observed value of some quantity $A$ over multiple samplings is given by the mean value with respect to the actual distribution, 
\[
\left\langle \left\langle A \right\rangle_\mu \right\rangle_{\{\mu\}} 
= \int \rmd {\mathbb P}\left( \left\{ \mu_i \right\} \middle| \left\{ \rho_i \right\} \right) \; \sum_{i=1}^K A_i \mu_i 
= \sum_{i=1}^K A_i \rho_i 
= \left\langle A \right\rangle_\rho .
\]
Similarly the variance over various realization is given by 
\begin{multline*}
\left\langle \left\langle A \right\rangle_\mu^2 \right\rangle_{\{\mu\}} - \left\langle \left\langle A \right\rangle_\mu \right\rangle_{\{\mu\}}^2  
= \int \rmd {\mathbb P}\left( \left\{ \mu_i \right\} \middle| \left\{ \rho_i \right\} \right) \; \left[ \sum_{i=1}^K A_i \mu_i \right]^2 
- \left[ \sum_{i=1}^K A_i \rho_i \right]^2 \\  
= \sum_{i \neq j} A_i A_j \underbrace{ \int \rmd {\mathbb P}\left( \left\{ \mu_i \right\} \middle| \left\{ \rho_i \right\} \right) \; \mu_i \mu_j }_{ = \rho_i \rho_j }  
+ \sum_{i=1}^K A_i^2 \int \rmd {\mathbb P}\left( \left\{ \mu_i \right\} \middle| \left\{ \rho_i \right\} \right) \; \mu_i^2  
- \left[ \sum_{i=1}^K A_i \rho_i \right]^2 \\ 
= \sum_{i=1}^K A_i^2 \int \rmd {\mathbb P}\left( \left\{ \mu_i \right\} \middle| \left\{ \rho_i \right\} \right) \; \left( \mu_i^2 - \rho_i^2 \right)
+ \underbrace{ \sum_{i, j = 1 }^K A_i A_j \rho_i \rho_j 
- \left[ \sum_{i=1}^K A_i \rho_i \right]^2 }_{ = 0 } \\  
= \frac{1}{N} \sum_{i=1}^K A_i^2 \rho_i 
= \frac{1}{N} \left\langle A^2 \right\rangle_\rho . 
\end{multline*}
Unsurprisingly, the more we sample the distribution the smaller the variance is. 
In the derivation we released the condition on the normalization of $\mu$ in order to have individual $\mu_i$ independent, 
however if the there is a large number of states (as in our case where we sample the distribution by the position), 
the constraint have very little influence as it enforces the value only on the last element. 
Moreover as the variance for each element is proportional to $1/N$ the distribution is well localized around optimum. 
However I don't have any exact proof nor estimate.

If we use it to estimate the variance of the heat caused by the sampling we get 
\[
\left\langle \left\langle {\mathcal Q} \right\rangle_\mu^2 \right\rangle_{\{\mu\}} - \left\langle \left\langle {\mathcal Q} \right\rangle_\mu \right\rangle_{\{\mu\}}^2  
= \frac{\Delta t}{\Delta T} \left\langle {\mathcal Q}^2 \right\rangle_\rho,  
\] 
where $\Delta t$ is the time-step and $\Delta T$ is the total time of simulation. 
In equilibrium the mean value of the heat is zero, hence we have
\[
\left\langle \left\langle {\mathcal Q} \right\rangle_\mu^2 \right\rangle_{\{\mu\}} - \left\langle \left\langle {\mathcal Q} \right\rangle_\mu \right\rangle_{\{\mu\}}^2 
\ge \frac{ D \Delta V^2 }{\ell^2 a (1-a) } \Delta t     
\]
in case only simple ratchet potential is considered. 
In more general case we have 
\[
\left\langle \left\langle {\mathcal Q} \right\rangle_\mu^2 \right\rangle_{\{\mu\}} - \left\langle \left\langle {\mathcal Q} \right\rangle_\mu \right\rangle_{\{\mu\}}^2 
\ge \left( \frac{1}{\gamma_M} + \frac{1}{\gamma_A} \right) k_B T \left\langle \nabla V^2 \right\rangle_\rho \Delta t .
\]

\end{document}
